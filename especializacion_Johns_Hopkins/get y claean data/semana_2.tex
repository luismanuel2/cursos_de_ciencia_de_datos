% Options for packages loaded elsewhere
\PassOptionsToPackage{unicode}{hyperref}
\PassOptionsToPackage{hyphens}{url}
%
\documentclass[
]{article}
\usepackage{amsmath,amssymb}
\usepackage{lmodern}
\usepackage{ifxetex,ifluatex}
\ifnum 0\ifxetex 1\fi\ifluatex 1\fi=0 % if pdftex
  \usepackage[T1]{fontenc}
  \usepackage[utf8]{inputenc}
  \usepackage{textcomp} % provide euro and other symbols
\else % if luatex or xetex
  \usepackage{unicode-math}
  \defaultfontfeatures{Scale=MatchLowercase}
  \defaultfontfeatures[\rmfamily]{Ligatures=TeX,Scale=1}
\fi
% Use upquote if available, for straight quotes in verbatim environments
\IfFileExists{upquote.sty}{\usepackage{upquote}}{}
\IfFileExists{microtype.sty}{% use microtype if available
  \usepackage[]{microtype}
  \UseMicrotypeSet[protrusion]{basicmath} % disable protrusion for tt fonts
}{}
\makeatletter
\@ifundefined{KOMAClassName}{% if non-KOMA class
  \IfFileExists{parskip.sty}{%
    \usepackage{parskip}
  }{% else
    \setlength{\parindent}{0pt}
    \setlength{\parskip}{6pt plus 2pt minus 1pt}}
}{% if KOMA class
  \KOMAoptions{parskip=half}}
\makeatother
\usepackage{xcolor}
\IfFileExists{xurl.sty}{\usepackage{xurl}}{} % add URL line breaks if available
\IfFileExists{bookmark.sty}{\usepackage{bookmark}}{\usepackage{hyperref}}
\hypersetup{
  pdftitle={semana 2},
  pdfauthor={Luis Ambrocio},
  hidelinks,
  pdfcreator={LaTeX via pandoc}}
\urlstyle{same} % disable monospaced font for URLs
\usepackage[margin=1in]{geometry}
\usepackage{color}
\usepackage{fancyvrb}
\newcommand{\VerbBar}{|}
\newcommand{\VERB}{\Verb[commandchars=\\\{\}]}
\DefineVerbatimEnvironment{Highlighting}{Verbatim}{commandchars=\\\{\}}
% Add ',fontsize=\small' for more characters per line
\usepackage{framed}
\definecolor{shadecolor}{RGB}{248,248,248}
\newenvironment{Shaded}{\begin{snugshade}}{\end{snugshade}}
\newcommand{\AlertTok}[1]{\textcolor[rgb]{0.94,0.16,0.16}{#1}}
\newcommand{\AnnotationTok}[1]{\textcolor[rgb]{0.56,0.35,0.01}{\textbf{\textit{#1}}}}
\newcommand{\AttributeTok}[1]{\textcolor[rgb]{0.77,0.63,0.00}{#1}}
\newcommand{\BaseNTok}[1]{\textcolor[rgb]{0.00,0.00,0.81}{#1}}
\newcommand{\BuiltInTok}[1]{#1}
\newcommand{\CharTok}[1]{\textcolor[rgb]{0.31,0.60,0.02}{#1}}
\newcommand{\CommentTok}[1]{\textcolor[rgb]{0.56,0.35,0.01}{\textit{#1}}}
\newcommand{\CommentVarTok}[1]{\textcolor[rgb]{0.56,0.35,0.01}{\textbf{\textit{#1}}}}
\newcommand{\ConstantTok}[1]{\textcolor[rgb]{0.00,0.00,0.00}{#1}}
\newcommand{\ControlFlowTok}[1]{\textcolor[rgb]{0.13,0.29,0.53}{\textbf{#1}}}
\newcommand{\DataTypeTok}[1]{\textcolor[rgb]{0.13,0.29,0.53}{#1}}
\newcommand{\DecValTok}[1]{\textcolor[rgb]{0.00,0.00,0.81}{#1}}
\newcommand{\DocumentationTok}[1]{\textcolor[rgb]{0.56,0.35,0.01}{\textbf{\textit{#1}}}}
\newcommand{\ErrorTok}[1]{\textcolor[rgb]{0.64,0.00,0.00}{\textbf{#1}}}
\newcommand{\ExtensionTok}[1]{#1}
\newcommand{\FloatTok}[1]{\textcolor[rgb]{0.00,0.00,0.81}{#1}}
\newcommand{\FunctionTok}[1]{\textcolor[rgb]{0.00,0.00,0.00}{#1}}
\newcommand{\ImportTok}[1]{#1}
\newcommand{\InformationTok}[1]{\textcolor[rgb]{0.56,0.35,0.01}{\textbf{\textit{#1}}}}
\newcommand{\KeywordTok}[1]{\textcolor[rgb]{0.13,0.29,0.53}{\textbf{#1}}}
\newcommand{\NormalTok}[1]{#1}
\newcommand{\OperatorTok}[1]{\textcolor[rgb]{0.81,0.36,0.00}{\textbf{#1}}}
\newcommand{\OtherTok}[1]{\textcolor[rgb]{0.56,0.35,0.01}{#1}}
\newcommand{\PreprocessorTok}[1]{\textcolor[rgb]{0.56,0.35,0.01}{\textit{#1}}}
\newcommand{\RegionMarkerTok}[1]{#1}
\newcommand{\SpecialCharTok}[1]{\textcolor[rgb]{0.00,0.00,0.00}{#1}}
\newcommand{\SpecialStringTok}[1]{\textcolor[rgb]{0.31,0.60,0.02}{#1}}
\newcommand{\StringTok}[1]{\textcolor[rgb]{0.31,0.60,0.02}{#1}}
\newcommand{\VariableTok}[1]{\textcolor[rgb]{0.00,0.00,0.00}{#1}}
\newcommand{\VerbatimStringTok}[1]{\textcolor[rgb]{0.31,0.60,0.02}{#1}}
\newcommand{\WarningTok}[1]{\textcolor[rgb]{0.56,0.35,0.01}{\textbf{\textit{#1}}}}
\usepackage{graphicx}
\makeatletter
\def\maxwidth{\ifdim\Gin@nat@width>\linewidth\linewidth\else\Gin@nat@width\fi}
\def\maxheight{\ifdim\Gin@nat@height>\textheight\textheight\else\Gin@nat@height\fi}
\makeatother
% Scale images if necessary, so that they will not overflow the page
% margins by default, and it is still possible to overwrite the defaults
% using explicit options in \includegraphics[width, height, ...]{}
\setkeys{Gin}{width=\maxwidth,height=\maxheight,keepaspectratio}
% Set default figure placement to htbp
\makeatletter
\def\fps@figure{htbp}
\makeatother
\setlength{\emergencystretch}{3em} % prevent overfull lines
\providecommand{\tightlist}{%
  \setlength{\itemsep}{0pt}\setlength{\parskip}{0pt}}
\setcounter{secnumdepth}{-\maxdimen} % remove section numbering
\ifluatex
  \usepackage{selnolig}  % disable illegal ligatures
\fi

\title{semana 2}
\author{Luis Ambrocio}
\date{}

\begin{document}
\maketitle

\tableofcontents

\hypertarget{leyendo-mysql}{%
\section{Leyendo MySQL}\label{leyendo-mysql}}

\begin{itemize}
\tightlist
\item
  Software de base de datos de código abierto gratuito y ampliamente
  utilizado
\item
  Ampliamente utilizado en aplicaciones basadas en Internet.
\item
  Los datos están estructurados en

  \begin{itemize}
  \tightlist
  \item
    Bases de datos
  \item
    Tablas dentro de bases de datos
  \item
    Campos dentro de tablas
  \end{itemize}
\item
  Cada fila se llama registro
\end{itemize}

\begin{center}\includegraphics[width=1\linewidth]{D:/luism/Documents/courses/assets/img/03_ObtainingData/database-schema} \end{center}

primero se installa MySQL

\url{http://dev.mysql.com/doc/refman/5.7/en/installing.html}

luego intsalar RMySQL

\begin{itemize}
\tightlist
\item
  Official instructions -
  \url{http://biostat.mc.vanderbilt.edu/wiki/Main/RMySQL} (may be useful
  for Mac/UNIX users as well)
\item
  Potentially useful guide -
  \url{http://www.ahschulz.de/2013/07/23/installing-rmysql-under-windows/}
\end{itemize}

conectar y enumerar bases de datos

\begin{center}\includegraphics[width=1\linewidth]{D:/luism/Pictures/Saved Pictures/mysql1} \end{center}

Conexión a hg19 y listas de tablas

\begin{center}\includegraphics[width=1\linewidth]{D:/luism/Pictures/Saved Pictures/mysql2} \end{center}

obteniendo dimenciones de una tabla especifica

\begin{center}\includegraphics[width=1\linewidth]{D:/luism/Pictures/Saved Pictures/mysql3} \end{center}

leyendo de la tabla

\begin{center}\includegraphics[width=1\linewidth]{D:/luism/Pictures/Saved Pictures/mysql4} \end{center}

seleccionando un subconjunto especifico

\begin{center}\includegraphics[width=1\linewidth]{D:/luism/Pictures/Saved Pictures/mysql5} \end{center}

cerrar la conexion

\begin{center}\includegraphics[width=1\linewidth]{D:/luism/Pictures/Saved Pictures/mysql6} \end{center}

\hypertarget{recursos-adicionales}{%
\subsection{Recursos adicionales}\label{recursos-adicionales}}

\begin{itemize}
\tightlist
\item
  Viñeta RMySQL
  \url{http://cran.r-project.org/web/packages/RMySQL/RMySQL.pdf}
\item
  Lista de comandos
  \url{http://www.pantz.org/software/mysql/mysqlcommands.html}

  \begin{itemize}
  \tightlist
  \item
    \textbf{No, no elimine, agregue o combine elementos de ensembl. Solo
    seleccione .}
  \item
    En general, tenga cuidado con los comandos mysql
  \end{itemize}
\item
  Una buena publicación de blog que resume algunos otros comandos
  \url{http://www.r-bloggers.com/mysql-and-r/}
\item
  \url{http://en.wikipedia.org/wiki/MySQL}
\item
  \url{http://www.mysql.com/}
\end{itemize}

\hypertarget{leyendo-hd5f}{%
\section{Leyendo HD5F}\label{leyendo-hd5f}}

HDF5

\begin{itemize}
\tightlist
\item
  Se utiliza para almacenar grandes conjuntos de datos.
\item
  Admite el almacenamiento de una variedad de tipos de datos
\item
  Formato de datos jerárquico
\item
  \emph{grupos} que contienen cero o más conjuntos de datos y metadatos

  \begin{itemize}
  \tightlist
  \item
    Tener un \emph{encabezado de grupo} con el nombre del grupo y una
    lista de atributos
  \item
    Tener una \emph{tabla de símbolos de grupo} con una lista de objetos
    en el grupo
  \end{itemize}
\item
  \emph{datasets} matriz multidimensional de elementos de datos con
  metadatos

  \begin{itemize}
  \tightlist
  \item
    Tener un \emph{encabezado} con nombre, tipo de datos, espacio de
    datos y diseño de almacenamiento
  \item
    Tener un \emph{data array} con los datos
  \end{itemize}
\end{itemize}

\url{http://www.hdfgroup.org/}

\begin{Shaded}
\begin{Highlighting}[]
\FunctionTok{invisible}\NormalTok{(}\ControlFlowTok{if}\NormalTok{(}\FunctionTok{file.exists}\NormalTok{(}\StringTok{"example.h5"}\NormalTok{))\{}\FunctionTok{file.remove}\NormalTok{(}\StringTok{"example.h5"}\NormalTok{)\})}
\end{Highlighting}
\end{Shaded}

\begin{Shaded}
\begin{Highlighting}[]
\FunctionTok{library}\NormalTok{(rhdf5)}
\NormalTok{file }\OtherTok{\textless{}{-}} \FunctionTok{h5createFile}\NormalTok{(}\StringTok{"example.h5"}\NormalTok{)}
\NormalTok{file}
\end{Highlighting}
\end{Shaded}

\begin{verbatim}
## [1] TRUE
\end{verbatim}

\begin{itemize}
\tightlist
\item
  Esto instalará paquetes de Bioconductor
  \url{http://bioconductor.org/}, que se usa principalmente para
  genómica pero también tiene buenos paquetes de ``big data''
\item
  Se puede utilizar para interactuar con conjuntos de datos hdf5.
\item
  Esta conferencia se basa muy de cerca en el tutorial de rhdf5 que se
  puede encontrar aquí
  \href{http://www.bioconductor.org/packages/release/bioc/\%20viñetas\%20/\%20rhdf5\%20/\%20inst\%20/\%20doc\%20/\%20rhdf5.pdf}{http://www.bioconductor.org/packages/release/bioc/vignettes/rhdf5/inst/doc/rhdf5.pdf}
\end{itemize}

crear grupos

\begin{Shaded}
\begin{Highlighting}[]
\NormalTok{created }\OtherTok{=} \FunctionTok{h5createGroup}\NormalTok{(}\StringTok{"example.h5"}\NormalTok{,}\StringTok{"foo"}\NormalTok{)}
\NormalTok{created }\OtherTok{=} \FunctionTok{h5createGroup}\NormalTok{(}\StringTok{"example.h5"}\NormalTok{,}\StringTok{"baa"}\NormalTok{)}
\NormalTok{created }\OtherTok{=} \FunctionTok{h5createGroup}\NormalTok{(}\StringTok{"example.h5"}\NormalTok{,}\StringTok{"foo/foobaa"}\NormalTok{)}
\FunctionTok{h5ls}\NormalTok{(}\StringTok{"example.h5"}\NormalTok{)}
\end{Highlighting}
\end{Shaded}

\begin{verbatim}
##   group   name     otype dclass dim
## 0     /    baa H5I_GROUP           
## 1     /    foo H5I_GROUP           
## 2  /foo foobaa H5I_GROUP
\end{verbatim}

Escribir a grupos

\begin{Shaded}
\begin{Highlighting}[]
\NormalTok{A }\OtherTok{=} \FunctionTok{matrix}\NormalTok{(}\DecValTok{1}\SpecialCharTok{:}\DecValTok{10}\NormalTok{,}\AttributeTok{nr=}\DecValTok{5}\NormalTok{,}\AttributeTok{nc=}\DecValTok{2}\NormalTok{)}
\FunctionTok{h5write}\NormalTok{(A, }\StringTok{"example.h5"}\NormalTok{,}\StringTok{"foo/A"}\NormalTok{)}
\NormalTok{B }\OtherTok{=} \FunctionTok{array}\NormalTok{(}\FunctionTok{seq}\NormalTok{(}\FloatTok{0.1}\NormalTok{,}\FloatTok{2.0}\NormalTok{,}\AttributeTok{by=}\FloatTok{0.1}\NormalTok{),}\AttributeTok{dim=}\FunctionTok{c}\NormalTok{(}\DecValTok{5}\NormalTok{,}\DecValTok{2}\NormalTok{,}\DecValTok{2}\NormalTok{))}
\FunctionTok{attr}\NormalTok{(B, }\StringTok{"scale"}\NormalTok{) }\OtherTok{\textless{}{-}} \StringTok{"liter"}
\FunctionTok{h5write}\NormalTok{(B, }\StringTok{"example.h5"}\NormalTok{,}\StringTok{"foo/foobaa/B"}\NormalTok{)}
\FunctionTok{h5ls}\NormalTok{(}\StringTok{"example.h5"}\NormalTok{)}
\end{Highlighting}
\end{Shaded}

\begin{verbatim}
##         group   name       otype  dclass       dim
## 0           /    baa   H5I_GROUP                  
## 1           /    foo   H5I_GROUP                  
## 2        /foo      A H5I_DATASET INTEGER     5 x 2
## 3        /foo foobaa   H5I_GROUP                  
## 4 /foo/foobaa      B H5I_DATASET   FLOAT 5 x 2 x 2
\end{verbatim}

escribir un conjunto de datos

\begin{Shaded}
\begin{Highlighting}[]
\NormalTok{df }\OtherTok{=} \FunctionTok{data.frame}\NormalTok{(1L}\SpecialCharTok{:}\NormalTok{5L,}\FunctionTok{seq}\NormalTok{(}\DecValTok{0}\NormalTok{,}\DecValTok{1}\NormalTok{,}\AttributeTok{length.out=}\DecValTok{5}\NormalTok{),}
  \FunctionTok{c}\NormalTok{(}\StringTok{"ab"}\NormalTok{,}\StringTok{"cde"}\NormalTok{,}\StringTok{"fghi"}\NormalTok{,}\StringTok{"a"}\NormalTok{,}\StringTok{"s"}\NormalTok{), }\AttributeTok{stringsAsFactors=}\ConstantTok{FALSE}\NormalTok{)}
\FunctionTok{h5write}\NormalTok{(df, }\StringTok{"example.h5"}\NormalTok{,}\StringTok{"df"}\NormalTok{)}
\FunctionTok{h5ls}\NormalTok{(}\StringTok{"example.h5"}\NormalTok{)}
\end{Highlighting}
\end{Shaded}

\begin{verbatim}
##         group   name       otype   dclass       dim
## 0           /    baa   H5I_GROUP                   
## 1           /     df H5I_DATASET COMPOUND         5
## 2           /    foo   H5I_GROUP                   
## 3        /foo      A H5I_DATASET  INTEGER     5 x 2
## 4        /foo foobaa   H5I_GROUP                   
## 5 /foo/foobaa      B H5I_DATASET    FLOAT 5 x 2 x 2
\end{verbatim}

leyendo datos

\begin{Shaded}
\begin{Highlighting}[]
\NormalTok{readA }\OtherTok{=} \FunctionTok{h5read}\NormalTok{(}\StringTok{"example.h5"}\NormalTok{,}\StringTok{"foo/A"}\NormalTok{)}
\NormalTok{readB }\OtherTok{=} \FunctionTok{h5read}\NormalTok{(}\StringTok{"example.h5"}\NormalTok{,}\StringTok{"foo/foobaa/B"}\NormalTok{)}
\NormalTok{readdf}\OtherTok{=} \FunctionTok{h5read}\NormalTok{(}\StringTok{"example.h5"}\NormalTok{,}\StringTok{"df"}\NormalTok{)}
\NormalTok{readA}
\end{Highlighting}
\end{Shaded}

\begin{verbatim}
##      [,1] [,2]
## [1,]    1    6
## [2,]    2    7
## [3,]    3    8
## [4,]    4    9
## [5,]    5   10
\end{verbatim}

Fragmentos de escritura y lectura

\begin{Shaded}
\begin{Highlighting}[]
\FunctionTok{h5write}\NormalTok{(}\FunctionTok{c}\NormalTok{(}\DecValTok{12}\NormalTok{,}\DecValTok{13}\NormalTok{,}\DecValTok{14}\NormalTok{),}\StringTok{"example.h5"}\NormalTok{,}\StringTok{"foo/A"}\NormalTok{,}\AttributeTok{index=}\FunctionTok{list}\NormalTok{(}\DecValTok{1}\SpecialCharTok{:}\DecValTok{3}\NormalTok{,}\DecValTok{1}\NormalTok{))}
\FunctionTok{h5read}\NormalTok{(}\StringTok{"example.h5"}\NormalTok{,}\StringTok{"foo/A"}\NormalTok{)}
\end{Highlighting}
\end{Shaded}

\begin{verbatim}
##      [,1] [,2]
## [1,]   12    6
## [2,]   13    7
## [3,]   14    8
## [4,]    4    9
## [5,]    5   10
\end{verbatim}

\hypertarget{notas-y-otros-recursos}{%
\subsection{Notas y otros recursos}\label{notas-y-otros-recursos}}

\begin{itemize}
\tightlist
\item
  hdf5 se puede utilizar para optimizar la lectura / escritura desde el
  disco en R
\item
  El tutorial de rhdf5:

  \begin{itemize}
  \tightlist
  \item
    \href{http://www.bioconductor.org/packages/release/bioc/vignettes/rhdf5\%20/inst/doc/rhdf5.pdf}{http://www.bioconductor.org/packages/release/bioc/vignettes/rhdf5/inst/doc/rhdf5.pdf}
  \end{itemize}
\item
  El grupo HDF tiene información sobre HDF5 en general
  \url{http://www.hdfgroup.org/HDF5/}
\end{itemize}

\hypertarget{leyendo-datos-de-la-web}{%
\section{Leyendo datos de la web}\label{leyendo-datos-de-la-web}}

Webscraping

\textbf{Webscraping}: Extracción de datos de forma programada del código
HTML de sitios web.

\begin{itemize}
\tightlist
\item
  Puede ser una excelente manera de obtener datos {[}Cómo Netflix hizo
  ingeniería inversa a Hollywood{]}
  (\url{http://www.theatlantic.com/technology/archive/2014/01/how-netflix-reverse-engineered-hollywood/282679/})
\item
  Muchos sitios web tienen información que puede querer leer
  programáticamente
\item
  En algunos casos, esto va en contra de los términos de servicio del
  sitio web.
\item
  Intentar leer demasiadas páginas demasiado rápido puede bloquear su
  dirección IP
\end{itemize}

ejemplo

\begin{center}\includegraphics[width=1\linewidth]{D:/luism/Documents/courses/assets/img/googlescholar} \end{center}

\url{http://scholar.google.com/citations?user=HI-I6C0AAAAJ\&hl=en}

\begin{Shaded}
\begin{Highlighting}[]
\NormalTok{con }\OtherTok{=} \FunctionTok{url}\NormalTok{(}\StringTok{"http://scholar.google.com/citations?user=HI{-}I6C0AAAAJ\&hl=en"}\NormalTok{)}
\NormalTok{htmlCode }\OtherTok{=} \FunctionTok{readLines}\NormalTok{(con)}
\FunctionTok{close}\NormalTok{(con)}
\end{Highlighting}
\end{Shaded}

obteniendo informacion con paquete httr

\begin{Shaded}
\begin{Highlighting}[]
\FunctionTok{library}\NormalTok{(httr)}
\FunctionTok{library}\NormalTok{(XML)}
\NormalTok{url }\OtherTok{\textless{}{-}} \StringTok{"http://scholar.google.com/citations?user=HI{-}I6C0AAAAJ\&hl=en"}
\NormalTok{html2 }\OtherTok{=} \FunctionTok{GET}\NormalTok{(url)}
\NormalTok{content2 }\OtherTok{=} \FunctionTok{content}\NormalTok{(html2,}\AttributeTok{as=}\StringTok{"text"}\NormalTok{)}
\NormalTok{parsedHtml }\OtherTok{=} \FunctionTok{htmlParse}\NormalTok{(content2,}\AttributeTok{asText=}\ConstantTok{TRUE}\NormalTok{)}
\FunctionTok{xpathSApply}\NormalTok{(parsedHtml, }\StringTok{"//title"}\NormalTok{, xmlValue)}
\end{Highlighting}
\end{Shaded}

\begin{verbatim}
## [1] "Jeff Leek - Google Scholar"
\end{verbatim}

accesando a sitios web con contraseña

\begin{Shaded}
\begin{Highlighting}[]
\NormalTok{pg1 }\OtherTok{=} \FunctionTok{GET}\NormalTok{(}\StringTok{"http://httpbin.org/basic{-}auth/user/passwd"}\NormalTok{)}
\NormalTok{pg1}
\end{Highlighting}
\end{Shaded}

\begin{verbatim}
## Response [http://httpbin.org/basic-auth/user/passwd]
##   Date: 2021-08-06 04:05
##   Status: 401
##   Content-Type: <unknown>
## <EMPTY BODY>
\end{verbatim}

\begin{Shaded}
\begin{Highlighting}[]
\NormalTok{pg2 }\OtherTok{=} \FunctionTok{GET}\NormalTok{(}\StringTok{"http://httpbin.org/basic{-}auth/user/passwd"}\NormalTok{,}
    \FunctionTok{authenticate}\NormalTok{(}\StringTok{"user"}\NormalTok{,}\StringTok{"passwd"}\NormalTok{))}
\NormalTok{pg2}
\end{Highlighting}
\end{Shaded}

\begin{verbatim}
## Response [http://httpbin.org/basic-auth/user/passwd]
##   Date: 2021-08-06 04:05
##   Status: 200
##   Content-Type: application/json
##   Size: 47 B
## {
##   "authenticated": true, 
##   "user": "user"
## }
\end{verbatim}

\begin{Shaded}
\begin{Highlighting}[]
\FunctionTok{names}\NormalTok{(pg2)}
\end{Highlighting}
\end{Shaded}

\begin{verbatim}
##  [1] "url"         "status_code" "headers"     "all_headers" "cookies"    
##  [6] "content"     "date"        "times"       "request"     "handle"
\end{verbatim}

usando handle

\begin{Shaded}
\begin{Highlighting}[]
\NormalTok{google }\OtherTok{=} \FunctionTok{handle}\NormalTok{(}\StringTok{"http://google.com"}\NormalTok{)}
\NormalTok{pg1 }\OtherTok{=} \FunctionTok{GET}\NormalTok{(}\AttributeTok{handle=}\NormalTok{google,}\AttributeTok{path=}\StringTok{"/"}\NormalTok{)}
\NormalTok{pg2 }\OtherTok{=} \FunctionTok{GET}\NormalTok{(}\AttributeTok{handle=}\NormalTok{google,}\AttributeTok{path=}\StringTok{"search"}\NormalTok{)}

\NormalTok{pg1}
\end{Highlighting}
\end{Shaded}

\begin{verbatim}
## Response [http://www.google.com/]
##   Date: 2021-08-06 04:05
##   Status: 200
##   Content-Type: text/html; charset=ISO-8859-1
##   Size: 14.1 kB
## <!doctype html><html itemscope="" itemtype="http://schema.org/WebPage" lang="...
## var f=this||self;var h,k=[];function l(a){for(var b;a&&(!a.getAttribute||!(b=...
## function n(a,b,c,d,g){var e="";c||-1!==b.search("&ei=")||(e="&ei="+l(d),-1===...
## google.y={};google.sy=[];google.x=function(a,b){if(a)var c=a.id;else{do c=Mat...
## document.documentElement.addEventListener("submit",function(b){var a;if(a=b.t...
## </style><style>body,td,a,p,.h{font-family:arial,sans-serif}body{margin:0;over...
## if (!iesg){document.f&&document.f.q.focus();document.gbqf&&document.gbqf.q.fo...
## }
## })();</script><div id="mngb"><div id=gbar><nobr><b class=gb1>Búsqueda</b> <a ...
## else top.location='/doodles/';};})();</script><input value="AINFCbYAAAAAYQzDE...
## ...
\end{verbatim}

\begin{Shaded}
\begin{Highlighting}[]
\NormalTok{pg2}
\end{Highlighting}
\end{Shaded}

\begin{verbatim}
## Response [http://www.google.com/webhp]
##   Date: 2021-08-06 04:05
##   Status: 200
##   Content-Type: text/html; charset=ISO-8859-1
##   Size: 14.1 kB
## <!doctype html><html itemscope="" itemtype="http://schema.org/WebPage" lang="...
## var f=this||self;var h,k=[];function l(a){for(var b;a&&(!a.getAttribute||!(b=...
## function n(a,b,c,d,g){var e="";c||-1!==b.search("&ei=")||(e="&ei="+l(d),-1===...
## google.y={};google.sy=[];google.x=function(a,b){if(a)var c=a.id;else{do c=Mat...
## document.documentElement.addEventListener("submit",function(b){var a;if(a=b.t...
## </style><style>body,td,a,p,.h{font-family:arial,sans-serif}body{margin:0;over...
## if (!iesg){document.f&&document.f.q.focus();document.gbqf&&document.gbqf.q.fo...
## }
## })();</script><div id="mngb"><div id=gbar><nobr><b class=gb1>Búsqueda</b> <a ...
## else top.location='/doodles/';};})();</script><input value="AINFCbYAAAAAYQzDE...
## ...
\end{verbatim}

\hypertarget{notas-y-otros-recursos-1}{%
\subsection{Notas y otros recursos}\label{notas-y-otros-recursos-1}}

\begin{itemize}
\tightlist
\item
  R Bloggers tiene varios ejemplos de raspado web
  \url{http://www.r-bloggers.com/?s=Web+Scraping}
\item
  El archivo de ayuda httr tiene ejemplos útiles
  \href{http://cran.r-project.org/web/packages/httr/\%20httr.pdf}{http://cran.r-project.org/web/packages/httr/httr.pdf}
\item
  Ver conferencias posteriores sobre API
\item
  \url{http://cran.r-project.org/web/packages/httr/httr.pdf}
\item
  \url{http://en.wikipedia.org/wiki/Web_scraping}
\end{itemize}

\hypertarget{leyendo-de-apis}{%
\section{Leyendo de APIs}\label{leyendo-de-apis}}

Application programming interfaces(Interfaces de programación de
aplicaciones)

\begin{center}\includegraphics[width=1\linewidth]{D:/luism/Documents/courses/assets/img/03_ObtainingData/twitter} \end{center}

crear una cuenta para ingresar a la API

\url{https://dev.twitter.com/apps}

luego crear una ``app'' y buscar los pasos para acceder a la api, el
ejemplo siguiente es de una cuenta ya creada para fines academicos

\begin{center}\includegraphics[width=1\linewidth]{D:/luism/Pictures/Saved Pictures/api1} \end{center}

convertir a JSON el objeto

\begin{center}\includegraphics[width=1\linewidth]{D:/luism/Pictures/Saved Pictures/api1} \end{center}

En general, mira la documentación.

\begin{itemize}
\tightlist
\item
  httr permite solicitudes \texttt{GET},\texttt{POST},
  \texttt{PUT},\texttt{DELETE} si está autorizado
\item
  Puede autenticarse con un nombre de usuario o una contraseña
\item
  La mayoría de las API modernas usan algo como oauth
\item
  httr funciona bien con Facebook, Google, Twitter, Githb, etc.
\end{itemize}

\hypertarget{reading-from-other-sources}{%
\section{Reading from other sources}\label{reading-from-other-sources}}

Hay un paquete para eso

\begin{itemize}
\tightlist
\item
  Roger tiene un buen video sobre cómo hay paquetes R para la mayoría de
  las cosas a las que querrá acceder.
\item
  Aquí voy a revisar brevemente algunos paquetes útiles
\item
  En general, la mejor forma de averiguar si el paquete R existe es el
  ``paquete R del mecanismo de almacenamiento de datos'' de Google

  \begin{itemize}
  \tightlist
  \item
    Por ejemplo: ``Paquete MySQL R''
  \end{itemize}
\end{itemize}

\emph{Interactuar más directamente con archivos}

\begin{itemize}
\tightlist
\item
  file: abre una conexión a un archivo de texto
\item
  url: abre una conexión a una URL
\item
  gzfile: abre una conexión a un archivo .gz
\item
  bzfile: abre una conexión a un archivo .bz2
\item
  \emph{?connections} para más información
\item
  Recuerde cerrar conexiones
\end{itemize}

\hypertarget{paquete-foreing}{%
\subsection{paquete foreing}\label{paquete-foreing}}

\begin{itemize}
\tightlist
\item
  Carga datos de Minitab, S, SAS, SPSS, Stata, Systat
\item
  Funciones básicas \emph{read.foo}

  \begin{itemize}
  \tightlist
  \item
    read.arff (Weka)
  \item
    read.dta (Stata)
  \item
    read.mtp (Minitab)
  \item
    read.octave (octava)
  \item
    read.spss (SPSS)
  \item
    read.xport (SAS)
  \end{itemize}
\item
  Consulte la página de ayuda para obtener más detalles.
\end{itemize}

\url{http://cran.r-project.org/web/packages/foreign/foreign.pdf}

\hypertarget{ejemplos-de-otros-paquetes-de-bases-de-datos}{%
\subsection{Ejemplos de otros paquetes de bases de
datos}\label{ejemplos-de-otros-paquetes-de-bases-de-datos}}

\begin{itemize}
\tightlist
\item
  RPostresSQL proporciona una conexión de base de datos compatible con
  DBI desde R. Tutorial- \url{https://code.google.com/p/rpostgresql/},
  archivo de ayuda-
  \url{http://cran.r-project.org/web/packages/RPostgreSQL/RPostgreSQL.pdf}
\item
  RODBC proporciona interfaces para múltiples bases de datos, incluidas
  PostgreQL, MySQL, Microsoft Access y SQLite. Tutorial -
  \href{http://cran.r-project.org/web/packages/RODBC/vignettes/RODBC.\%20pdf}{http://cran.r-project.org/web/packages/RODBC/vignettes/RODBC.pdf},
  archivo de ayuda -
  \href{http://cran.r-project.org/web/packages/RODBC/RODBC.\%20pdf}{http://cran.r-project.org/web/packages/RODBC/RODBC.pdf}
\item
  RMongo \url{http://cran.r-project.org/web/packages/RMongo/RMongo.pdf}
  (ejemplo de Rmongo \url{http://www.r-bloggers.com/r-and-mongodb/}) y
  \href{http:\%20//cran.r-project.org/web/packages/rmongodb/rmongodb.pdf}{rmongodb}
  proporcionan interfaces a MongoDb.
\end{itemize}

\hypertarget{leer-imuxe1genes}{%
\subsection{Leer imágenes}\label{leer-imuxe1genes}}

\begin{itemize}
\tightlist
\item
  jpeg - \url{http://cran.r-project.org/web/packages/jpeg/index.html}
\item
  readbitmap -
  \url{http://cran.r-project.org/web/packages/readbitmap/index.html}
\item
  png - \url{http://cran.r-project.org/web/packages/png/index.html}
\item
  EBImage (Bioconductor) -
  \href{http://www.bioconductor.org/packages/2.13/bioc/html/EBImage.\%20html}{http://www.bioconductor.org/packages/2.13/bioc/html/EBImage.html}
\end{itemize}

\hypertarget{lectura-de-datos-gis}{%
\subsection{Lectura de datos GIS}\label{lectura-de-datos-gis}}

\begin{itemize}
\tightlist
\item
  rgdal - \url{http://cran.r-project.org/web/packages/rgdal/index.html}
\item
  rgeos - \url{http://cran.r-project.org/web/packages/rgeos/index.html}
\item
  raster -
  \url{http://cran.r-project.org/web/packages/raster/index.html}
\end{itemize}

\hypertarget{leer-datos-musicales}{%
\subsection{Leer datos musicales}\label{leer-datos-musicales}}

\begin{itemize}
\tightlist
\item
  tuneR - \url{http://cran.r-project.org/web/packages/tuneR/}
\item
  seewave - \url{http://rug.mnhn.fr/seewave/}
\end{itemize}

\end{document}
